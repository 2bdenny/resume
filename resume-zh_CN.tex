% !TEX TS-program = xelatex
% !TEX encoding = UTF-8 Unicode
% !Mode:: "TeX:UTF-8"

\documentclass{resume}
\usepackage{zh_CN-Adobefonts_external} % Simplified Chinese Support using external fonts (./fonts/zh_CN-Adobe/)
% \usepackage{NotoSansSC_external}
% \usepackage{NotoSerifCJKsc_external}
% \usepackage{zh_CN-Adobefonts_internal} % Simplified Chinese Support using system fonts
\usepackage{linespacing_fix} % disable extra space before next section
\usepackage{cite}
\usepackage{geometry}
\geometry{top=1.0cm, bottom=0.5cm}
\begin{document}
\pagenumbering{gobble} % suppress displaying page number
\name{沈宇桔}
\basicInfo{
  \email{denny.syj@hotmail.com} \textperiodcentered\
  \phone{(+86)15861812223} \textperiodcentered\
  \homepage{http://moon.nju.edu.cn/\textasciitilde yuju/}
}

\section{\faGraduationCap\  教育背景}
\datedsubsection{\textbf{南京大学}, 江苏}{2016年9月 -- 2019年6月}
\textit{硕士}\ 计算机科学与技术系 (计算机软件研究所)
\datedsubsection{\textbf{南京大学}, 江苏}{2012年9月 -- 2016年6月}
\textit{学士}\ 计算机科学与技术系

\section{\faStar\ 求职意向}
\role{江浙沪}

\section{\faUsers\ 实习经历}
\datedsubsection{\textbf{腾讯, TEG}, 深圳}{2018年6月 -- 2018年8月}
\role{实习}{后台开发}
\begin{itemize}
  \item PCDN客户端代码稳定性测试和测试覆盖率优化
  \item PCDN客户端批量升级脚本
  \item 实习生项目比特币自动交易系统前端和组织测试
\end{itemize}

\section{\faSearch\ 项目经历}
\datedsubsection{\textbf{ReScue}}{2017年6月 -- 2018年4月}
\role{研究生阶段工作}{ASE '18, CCF-A, 一作, ACM SIGSOFT Distinguished PAPER AWARD~\cite{shen2018rescue}}
正则表达式DoS (ReDoS) 漏洞检测工具, 已开源, https://2bdenny.github.io/ReScue/
\begin{itemize}
  \item 支持多种语言 (js, python, php) 的正则表达式提取工具
  \item 基于遗传算法实现ReDoS漏洞自动检测工具
  \item 正则表达式匹配模拟工具
\end{itemize}

\datedsubsection{\textbf{BungoHelper}}{2017年4月}
\role{个人项目}{Chrome插件}
网页游戏文豪与炼金术士的Chrome提示插件, https://github.com/2bdenny/Bungo\_Helper
\begin{itemize}
  \item 截获游戏服务器的response, 提取隐藏信息并展示给玩家
\end{itemize}

\datedsubsection{\textbf{MR.Hunter}}{2015年11月 -- 2016年11月}
\role{研究生阶段工作}{Python, Web}
蜕变测试Demo (蜕变测试理论上可用于机器学习程序, 编译器等复杂程序的自动化测试)
\begin{itemize}
  \item 在线蜕变测试代码提交和自动验证系统
  \item 支持用户管理和积分排名, 提供简单的测试代码查重功能
\end{itemize}

\section{\faCogs\ 个人技能}
\begin{itemize}[parsep=0.5ex]
  \item IT技能: Java > C/C++ == Python, 熟悉常用的数据结构, 算法和设计模式
  \item 平台: Linux, 作为日常开发环境使用
  \item 英语: CET-6, 良好的读写能力
\end{itemize}

%\newpage
% \bibliographystyle{ACM-Reference-Format}
\bibliographystyle{IEEEtran}
\bibliography{mycite}
\end{document}
